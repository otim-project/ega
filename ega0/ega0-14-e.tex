\section*{Summary}
\label{section:ega0_IV-summary}

\oldpage[IV-1]{5}

\begin{longtable}{ll}
  \textsection14. & Combinatorial dimension of a topological space.\\
  \textsection15. & $M$-regular sequences and $\sh{F}$-regular sequences.\\
  \textsection16. & Dimension and depth of Noetherian local rings.\\
  \textsection17. & Regular rings.\\
  \textsection18. & Supplement on extensions of algebras.\\
  \textsection19. & Formally smooth algebras and Cohen rings.\\
  \textsection20. & Derivations and differentials.\\
  \textsection21. & Differentials in rings of characteristic $p$.\\
  \textsection22. & Differential criteria for smoothness and regularity.\\
  \textsection23. & Japanese rings.\\
\end{longtable}
\bigskip

Almost all of the preceding sections have been focused on the exposition of ideas of commutative algebra that will be used throughout Chapter~IV.
Even though a large amount of these ideas already appear in multiple works (\cite{I-1,I-12,I-13,I-17,IV-30}), we thought that it would be more practical for the reader to have a coherent, vaguely independent exposition.
Together with \textsection\textsection5, 6, and 7 of Chapter~IV (where we use the language of schemes), these sections constitute, in the middle of our treatise, a miniature special treatise, somewhat independent of Chapters~I to III, and one that aims to present, in a coherent manner, the properties of rings that ``behave well'' relative to operations such as completion, or integral closure, by systematically associating these properties to more general ideas.\footnote{The majority of properties which we discuss were discovered by Chevalley, Zariski, Nagata, and Serre. The method used here was first developed in the autumn of 1961, in a course taught a Harvard University by A.~Grothendieck.}
