\cite{I-1}.

\section{Affine morphisms}
\label{section:affine-morphisms}

\subsection{$S$-preschemes and $\mathcal{O}_S$-algebras}
\label{subsection:s-preschemes-algebras}

\begin{env}[1.1.1]
\label{2.1.1.1}
Let $S$ be a prescheme, $X$ an $S$-prescheme, and $f:X\to S$ its structure morphism.
We know \sref[0]{0.4.2.4} that the direct image $f_*(\OO_X)$ is an $\OO_S$-algebra, which we
\oldpage[II]{6}
denote $\sh{A}(X)$ when there is little chance of confusion; if $U$ is an open subset of $S$, then we have
\[
  \sh{A}(f^{-1}(U))=\sh{A}(X)|U.
\]
Similarly, for every $\OO_X$-module $\sh{F}$ (resp. every $\OO_X$-algebra $\sh{B}$), we write $\sh{A}(\sh{F})$ (resp. $\sh{A}(\sh{B})$) for the direct image $f_*(\sh{F})$ (resp. $f_*(\sh{B})$) which is an $\sh{A}(X)$-module (resp. an $\sh{A}(X)$-algebra) and not only an $\OO_S$-module (resp. an $\OO_S$-algebra).
\end{env}

\begin{env}[1.1.2]
\label{2.1.1.2}
Let $Y$ be a second $S$-prescheme, $g:Y\to S$ its structure morphism, and $h:X\to Y$ an $S$-morphism; we then have the commutative diagram
\[
\label{2.1.1.2.1}
  \xymatrix{
    X\ar[rr]^h\ar[rd]_f & &
    Y\ar[ld]^g\\
    & S.
  }
  \tag{1.1.2.1}
\]

We have by definition $h=(\psi,\theta)$, where $\theta:\OO_Y\to h_*(\OO_X)=\psi_*(\OO_X)$ is a homomorphism of sheaves of rings; we induce \sref[0]{0.4.2.2} a homomorphism of $\OO_S$-algebras $g_*(\theta):g_*(\OO_Y)\to g_*(h_*(\OO_X))=f_*(\OO_X)$, in other words, a homomorphism of $\OO_S$-algebras $\sh{A}(Y)\to\sh{A}(X)$, which we denote by $\sh{A}(h)$.
If $h':Y\to Z$ is a second $S$-morphism, then it is immediate that $\sh{A}(h'\circ h)=\sh{A}(h)\circ\sh{A}(h')$.
We havve thus define a \emph{contravariant functor $\sh{A}(X)$} from the category of $S$-preschemes to the category of $\OO_S$-algebras.

Now let $\sh{F}$ be an $\OO_X$-module, $\sh{G}$ an $\OO_Y$-module, and $u:\sh{G}\to\sh{F}$ an $h$-morphism, that is \sref[0]{0.4.4.1} a homomorphism of $\OO_Y$-modules $\sh{G}\to h_*(\sh{F})$.
Then $g_*(u):g_*(\sh{G})\to g_*(h_*(\sh{F}))=f_*(\sh{F})$ is a homomorphism $\sh{A}(\sh{G})\to\sh{A}(\sh{F})$ of $\OO_S$-modules, which we denote by $\sh{A}(u)$; in addition, the pair $(\sh{A}(h),\sh{A}(u))$ form a \emph{di-homomorphism} from the $\sh{A}(Y)$-module $\sh{A}(\sh{G})$ to the $\sh{A}(X)$-module $\sh{A}(\sh{F})$.
\end{env}

\begin{env}[1.1.3]
\label{2.1.1.3}
If we fix the prescheme $S$, then we can consider the pairs $(X,\sh{F})$, where $X$ is an $S$-prescheme and $\sh{F}$ is an $\OO_X$-module, as forming a \emph{category}, by defining a \emph{morphism} $(X,\sh{F})\to(Y,\sh{G})$ as a pair $(h,u)$, where $h:X\to Y$ is an $S$-morphism and $u:\sh{G}\to\sh{F}$ is an $h$-morphism.
We can theen say that $(\sh{A}(X),\sh{A}(\sh{F}))$ is a \emph{contravariant functor} with values in the category whose objects are pairs consisting of an $\OO_S$-algebra and a module over that algebra, and the morphisms are the di-homomorphisms.
\end{env}

\subsection{Affine preschemes over a prescheme}
\label{subsection:affine-preschemes-over-a-prescheme}

\begin{defn}[1.2.1]
\label{2.1.2.1}
Let $X$ be an $S$-prescheme, $f:X\to S$ its structure morphism.
We say that $X$ is \emph{affine over $S$} if there exists a cover $(S_\alpha)$ of $S$ by affine open sets such that for all $\alpha$, the induced prescheme on $X$ by the open set $f^{-1}(S_\alpha)$ is affine.
\end{defn}

\begin{exm}[1.2.2]
\label{2.1.2.2}
Every closed subprescheme of $S$ is an affine $S$-prescheme over $S$ (\sref[I]{1.4.2.3} and \sref[I]{1.4.2.4}).
\end{exm}

\begin{rmk}[1.2.3]
\label{2.1.2.3}
An affine prescheme $X$ over $S$ is not necessarily an affine scheme, as the example $X=S$ shows \sref{2.1.2.2}.
On the other hand, if an affine scheme $X$ is an $S$-prescheme, then $X$ is not necessarily affine over
\oldpage[II]{7}
$S$ (see Example~\sref{2.1.3.3}).
However, remember that if $S$ is a \emph{scheme}, then every $S$-prescheme which is an affine scheme is affine over $S$ \sref[I]{1.5.5.10}.
\end{rmk}

\begin{prop}[1.2.4]
\label{2.1.2.4}
Every $S$-prescheme which is affine over $S$ is separated over $S$ (in other words, it is an $S$-scheme).
\end{prop}

\begin{proof}
\label{proof-2.1.2.4}
This follows immediately from \sref[I]{1.5.5.5} and \sref[I]{1.5.5.8}.
\end{proof}

\begin{prop}[1.2.5]
\label{2.1.2.5}
Let $X$ be an $S$-scheme affine over $S$, $f:X\to S$ its structure morphism.
For every open $U\subset S$, $f^{-1}(U)$ is affine over $U$.
\end{prop}

\begin{proof}
\label{proof-2.1.2.5}
By Definition~\sref{2.1.2.1}, we can reduce to the case where $S=\Spec(A)$ and $X=\Spec(B)$ are affine; then $f=({}^a\vphi,\wt{\vphi})$, where $\vphi:A\to B$ is a homomorphism.
As the $D(g)$ for $g\in A$ form a basis for $S$, we reduce to the case where $U=D(g)$; but we then know that $f^{-1}(U)=D(\vphi(g))$ (\textbf{I},~1.2.2.2), hence the proposition.
\end{proof}

\begin{prop}[1.2.6]
\label{2.1.2.6}
Let $X$ be an $S$-scheme affine over $S$, $f:X\to S$ its structure morphism.
For every quasi-coherent $\OO_X$-module $\sh{F}$, $f_*(\sh{F})$ is a quasi-coherent $\OO_S$-module.
\end{prop}

\begin{proof}
\label{proof-2.1.2.6}
Taking into account Proposition~\sref{2.1.2.4}, this follows from \sref[I]{1.9.2.2}[a].
\end{proof}

In particular, the $\OO_S$-algebra $\sh{A}(X)=f_*(\OO_X)$ is \emph{quasi-coherent}.

\begin{prop}[1.2.7]
\label{2.1.2.7}
Let $X$ be an $S$-scheme affine over $S$.
For every $S$-prescheme $Y$, the map $h\mapsto\sh{A}(h)$ from the set $\Hom_S(Y,X)$ to the set $\Hom(\sh{A}(X),\sh{A}(Y))$ \sref{2.1.1.2} is bijective.
\end{prop}

\begin{proof}
\label{proof-2.1.2.7}
Let $f:X\to S$ and $g:Y\to S$ be the structure morphisms.
First, suppose that $S=\Spec(A)$ and $X=\Spec(B)$ are affine; we must prove that for every homomorphism $\omega:f_*(\OO_X)\to g_*(\OO_Y)$ of $\OO_S$-algebras, there exists a unique $S$-morphism $h:Y\to X$ such that $\sh{A}(h)=\omega$.
By definition, for every open $U\subset S$, $\omega$ defines a homomorphism $\omega_U=\Gamma(U,\omega):\Gamma(f^{-1}(U),\OO_X)\to\Gamma(g^{-1}(U),\OO_Y)$ of $\Gamma(U,\OO_S)$-algebras.
In particular, for $U=S$, this gives a homomorphism $\vphi:\Gamma(X,\OO_X)\to\Gamma(Y,\OO_Y)$ of $\Gamma(S,\OO_S)$-algebras, to which corresponds a well-defined $S$-morphism $h:Y\to X$, since $X$ is affine \sref[I]{1.2.2.4}.
It remains to prove that $\sh{A}(h)=\omega$, in other words, for every open set $U$ of a basis for $S$, $\omega_U$ coincides with the homomorphism of algebras $\vphi_U$ corresponding to the $S$-morphism $g^{-1}(U)\to f^{-1}(U)$, a restiction of $h$.
We can reduce to the case where $U=D(\lambda)$, with $\lambda\in S$; then, if $f=({}^a\rho,\wt{\rho})$, where $\rho:A\to B$ is a ring homomorphism, we have $f^{-1}(U)=D(\mu)$, where $\mu=\rho(\lambda)$, and $\Gamma(f^{-1}(U),\OO_X)$ is the ring of fractions $B_\mu$; the diagram
\[
  \xymatrix{
    B\ar[r]^\vphi\ar[d] &
    \Gamma(Y,\OO_Y)\ar[d]\\
    B_\mu\ar[r]^{\vphi_U} &
    \Gamma(g^{-1}(U),\OO_Y)
  }
\]
is commutative, and so is the analogous diagram where $\vphi_U$ is replaced by $\omega_U$; the equality $\vphi_U=\omega_U$ then follows from the universal property of rings of fractions \sref[0]{0.1.2.4}.

We now pass to the general case; let $(S_\alpha)$ be a cover of $S$ by affine open sets
\oldpage[II]{8}
such that the $f^{-1}(S_\alpha)$ are affine.
Then every homomorphism $\omega:\sh{A}(X)\to\sh{A}(Y)$ of $\OO_S$-algebras gives by restriction a family of homomorphisms
\[
  \omega_\alpha:\sh{A}(f^{-1}(S_\alpha))\to\sh{A}(g^{-1}(S_\alpha))
\]
of $\OO_{S_\alpha}$-algebras, hence a family of $S_\alpha$-morphisms $h_\alpha:g^{-1}(S_\alpha)\to f^{-1}(S_\alpha)$ by the above.
It remains to see that for every affine open set of a basis for $S_\alpha\cap S_\beta$, the restriction of $h_\alpha$ and $h_\beta$ to $g^{-1}(U)$ coincide, which is evident since by the above, these restrictions both correspond to the homomorphism $\sh{A}(X)|U\to\sh{A}(Y)|U$, a restriction of $\omega$.
\end{proof}

\begin{cor}[1.2.8]
\label{2.1.2.8}
Let $X$ and $Y$ be two $S$-schemes which are affine over $S$.
For an $S$-morphism $h:Y\to X$ to be an isomorphism, it is necessary and sufficient for $\sh{A}(h):\sh{A}(X)\to\sh{A}(Y)$ to be an isomorphism.
\end{cor}

\begin{proof}
\label{proof-2.1.2.8}
This follows immediately from Proposition~\sref{2.1.2.7} and from the functorial nature of $\sh{A}(X)$.
\end{proof}

\subsection{Affine preschemes over $S$ associated to an $\mathcal{O}_S$-algebra}
\label{subsection:affine-preschemes-associated-to-algebra}

\begin{prop}[1.3.1]
\label{2.1.3.1}
Let $S$ be a prescheme.
For every quasi-coherent $\OO_S$-algebra $\sh{B}$, there exists a prescheme $X$ affine over $S$, defined up to unique $S$-isomorphism, such that $\sh{A}(X)=\sh{B}$.
\end{prop}

\begin{proof}
\label{proof-2.1.3.1}
The uniqueness follows from Corollary~\sref{2.1.2.8}; we prove the existence of $X$.
For every affine open $U\subset S$, let $X_U$ be the prescheme $\Spec(\Gamma(U,\sh{B}))$; as $\Gamma(U,\sh{B})$ is a $\Gamma(U,\OO_S)$-algebras, $X_U$ is an $S$-prescheme \sref[I]{1.1.6.1}.
In addition, as $\sh{B}$ is quasi-coherent, the $\OO_S$-algebra $\sh{A}(X_U)$ canonically identifies with $\sh{B}|U$ (\sref[I]{1.1.3.7}, \sref[I]{1.1.3.13}, \sref[I]{1.1.6.3}).
Let $V$ be a second affine open subset of $S$, and let $X_{U,V}$ be the prescheme induced by $X_U$ on $f_U^{-1}(U\cap V)$, where $f_U$ denotes the structure morphism $X_U\to S$; $X_{U,V}$ and $X_{V,U}$ are affine over $U\cap V$ \sref{2.1.2.5}, and by definition $\sh{A}(X_{U,V})$ and $\sh{A}(X_{V,U})$ canonically identify with $\sh{B}|(U\cap V)$.
Hence there is \sref{2.1.2.8} a canonical $S$-isomorphism $\theta_{U,V}:X_{U,V}\to X_{U,V}$; in addition, if $W$ is a third affine open subset of $S$, and if $\theta_{U,V}'$, $\theta_{V,W}'$, and $\theta_{U,W}'$ are the restrictions of $\theta_{U,V}$, $\theta_{V,W}$, and $\theta_{U,W}$ to the inverse images of $U\cap V\cap W$ in $X_V$, $X_W$, and $X_W$ respectively under the structure morphisms, then we have $\theta_{U,V}'\circ\theta_{V,W}'=\theta_{U,W}'$.
As a result, there exists a prescheme $X$, a cover $(T_U)$ of $X$ by affine open sets, and for every $U$ an isomorphism $\vphi_U:X_U\to T_U$, such that $\vphi_U$ maps $f_U^{-1}(U\cap V)$ to $T_U\cap T_V$, and we have $\theta_{U,V}=\vphi_U^{-1}\circ\vphi_V$ \sref[I]{1.2.3.1}.
The morphism $g_U=f_U\circ\vphi_U^{-1}$ makes $T_U$ an $S$-prescheme, and the morphisms $g_U$ and $g_V$ coincide on $T_U\cap T_V$, hence $X$ is an $S$-prescheme.
In addition, it is clear by definition that $X$ is affine over $S$ and that $\sh{A}(T_U)=\sh{B}|U$, hence $\sh{A}(X)=\sh{B}$.
\end{proof}

We say that the $S$-scheme $X$ defined in this way is \emph{associated to the $\OO_S$-algebra $\sh{B}$}, or is the \emph{spectrum of $\sh{B}$}, and we denote it by $\Spec(\sh{B})$.

\begin{cor}[1.3.2]
\label{2.1.3.2}
Let $X$ be a prescheme affine over $S$, $f:X\to S$ the structure morphism.
For every affine open $U\subset S$, the induced prescheme on $f^{-1}(U)$ is the affine scheme with ring $\Gamma(U,\sh{A}(X))$.
\end{cor}

\begin{proof}
\label{proof-2.1.3.2}
\oldpage[II]{9}
As we can suppose that $X$ is associated to an $\OO_S$-algebra by Propositions~\sref{2.1.2.6} and \sref{2.1.3.1}, the corollary follows from the construction of $X$ described in Proposition~\sref{2.1.3.1}.
\end{proof}

\begin{exm}[1.3.3]
\label{2.1.3.3}
Let $S$ be the affine plane over a field $K$, where the point $0$ has been doubled \sref[I]{1.5.5.11}; with the notation of \sref[I]{1.5.5.11}, $S$ is the union of two affine open sets $Y_1$ and $Y_2$; if $f$ is the open immersion $Y_1\to S$, then $f^{-1}(Y_2)=Y_1\cap Y_2$ is not an affine open set in $Y_1$ (\emph{loc. cit.}), hence we have an example of an affine scheme which is not affine over $S$.
\end{exm}

\begin{cor}[1.3.4]
\label{2.1.3.4}
Let $S$ be an affine scheme; for an $S$-prescheme $X$ to be affine over $S$, it is necessary and sufficient for $X$ to be an affine scheme.
\end{cor}

\begin{cor}[1.3.5]
\label{2.1.3.5}
Let $X$ be a prescheme affine over a prescheme $S$, and let $Y$ be an $X$-prescheme.
For $Y$ to be affine over $S$, it is necessary and sufficient for $Y$ to be affine over $X$.
\end{cor}

\begin{proof}
\label{proof-2.1.3.5}
We immediately reduce to the case where $S$ is an affine scheme, and then we can reduce to the case where $X$ is an affine scheme \sref{2.1.3.4}; the two conditions of the statement then give that $Y$ is an affine scheme \sref{2.1.3.4}.
\end{proof}

\begin{env}[1.3.6]
\label{2.1.3.6}
Let $X$ be a prescheme affine over $S$.
To define a prescheme $Y$ affine \emph{over $X$}, it is equivalent, by Corollary~\sref{2.1.3.5}, to give a prescheme $Y$ affine \emph{over $S$}, and an $S$-morphism $g:Y\to X$; in other words (Proposition~\sref{2.1.3.1} and \sref{2.1.2.7}), it is equivalent to give a quasi-coherent $\OO_S$-algebra $\sh{B}$ and a homomorphism $\sh{A}(X)\to\sh{B}$ of $\OO_S$-algebras (which can be considered as defining on $\sh{B}$ an $\sh{A}(X)$-algebra structure).
If $f:X\to S$ is the structure morphism, then we have $\sh{B}=f_*(g_*(\OO_Y))$.
\end{env}

\begin{cor}[1.3.7]
\label{2.1.3.7}
Let $X$ be a prescheme affine over $S$; for $X$ to be of finite type over $S$, it is necessary and sufficient for the quasi-coherent $\OO_S$-algebra $\sh{A}(X)$ to be of finite type \sref[I]{1.9.6.2}.
\end{cor}

\begin{proof}
\label{proof-2.1.3.7}
By definition \sref[I]{1.9.6.2}, we can reduce to the case where $S$ is affine; then $X$ is an affine scheme \sref{2.1.3.4}, and if $S=\Spec(A)$, $X=\Spec(B)$, then $\sh{A}(X)$ is the $\OO_S$-algebra $\wt{B}$; as $\Gamma(U,\wt{B})=B$, the corollary follows from \sref[I]{1.9.6.2} and \sref[I]{1.6.3.3}.
\end{proof}

\begin{cor}[1.3.8]
\label{2.1.3.8}
Let $X$ be a prescheme affine over $S$; for $X$ to be reduced, it is necessary and sufficient for the quasi-coherent $\OO_X$-algebra $\sh{A}(X)$ to be reduced \sref[0]{0.4.1.4}.
\end{cor}

\begin{proof}
\label{proof-2.1.3.8}
The question is local on $S$; by Corollary~\sref{2.1.3.2}, the corollary follows from \sref[I]{1.5.1.1} and \sref[I]{1.5.1.4}.
\end{proof}

\subsection{Quasi-coherent sheaves over a prescheme affine over $S$}
\label{subsection:quasi-coherent-sheaves-on-prescheme-affine-over}

\begin{prop}[1.4.1]
\label{2.1.4.1}
Let $X$ be a prescheme affine over $S$, $Y$ an $S$-prescheme, and $\sh{F}$ (resp.~$\sh{G}$) a quasi-coherent $\OO_X$-module (resp.~an $\OO_Y$-module).
Then the map $(h,u)\mapsto(\sh{A}(h),\sh{A}(u))$ from the set of morphism $(Y,\sh{G})\to(X,\sh{F})$ to the set of di-homomorphisms $(\sh{A}(X),\sh{A}(\sh{F}))\to(\sh{A}(Y),\sh{A}(\sh{G}))$ (\sref{2.1.1.2} and \sref{2.1.1.3}) is bijective.
\end{prop}

\begin{proof}
\label{proof-2.1.4.1}
The proof follows exactly as that of Proposition~\sref{2.1.2.7} by using \sref[I]{1.2.2.5} and \sref[I]{1.2.2.4}, and the details are left to the reader.
\end{proof}

\begin{cor}[1.4.2]
\label{2.1.4.2}
If, in addition to the hypotheses of Proposition~\sref{2.1.4.1}, we suppose that $Y$ is affine over $S$, then for $(h,u)$ to be an isomorphism, it is necessary and sufficient for $(\sh{A}(h),\sh{A}(u))$ to be a di-isomorphism.
\end{cor}

\begin{prop}[1.4.3]
\label{2.1.4.3}
\oldpage{10}
For every pair $(\sh{B},\sh{M})$ consisting of a quasi-coherent $\OO_S$-algebra $\sh{B}$ and a quasi-coherent $\sh{B}$-module $\sh{M}$ \emph{(considered as an $\OO_S$-module or as a $\sh{B}$-module, which are equivalent~\sref[I]{1.9.6.1})}, there exists a pair $(X,\sh{F})$ consisting of a prescheme $X$ affine over $S$ and of a quasi-coherent $\OO_X$-module $\sh{F}$, such that $\sh{A}(X)=\sh{B}$ and $\sh{A}(\sh{F})=\sh{M}$; in addition, this couple is determined up to unique isomorphism.
\end{prop}

\begin{proof}
\label{proof-2.1.4.3}
The uniqueness follows from Proposition~\sref{2.1.4.1} and Corollary~\sref{2.1.4.2}; the existence is proved as in Proposition~\sref{2.1.3.1}, and we leave the details to the reader.
\end{proof}

We denote by $\wt{\sh{M}}$ the $\OO_X$-module $\sh{F}$, and we say that it is \emph{associated} to the quasi-coherent $\sh{B}$-module $\sh{M}$; for every affine open $U\subset S$, $\sh{M}|p^{-1}(U)$ (where $p$ is the structure morphism $X\to S$) is canonically isomorphic to $(\Gamma(U,\sh{M}))^\sim$.

\begin{cor}[1.4.4]
\label{2.1.4.4}
On the category of quasi-coherent $\sh{B}$-modules, $\wt{\sh{M}}$ is an additive covariant exact functor in $\sh{M}$, which commutes with inductive limitd and direct sums.
\end{cor}

\begin{proof}
\label{proof-2.1.4.4}
We immediately reduce to the case where $S$ is affine, and the corollary then follows from \sref[I]{1.1.3.5}, \sref[I]{1.1.3.9}, and \sref[I]{1.1.3.11}.
\end{proof}

\begin{cor}[1.4.5]
\label{2.1.4.5}
Under the hypotheses of Proposition~\sref{2.1.4.3}, for $\wt{\sh{M}}$ to be an $\OO_X$-module of finite type, it is necessary and sufficient for $\sh{M}$ to be a $\sh{B}$-module of finite type.
\end{cor}

\begin{proof}
\label{proof-2.1.4.5}
We immediately reduce to the case where $S=\Spec(A)$ is an affine scheme.
Then $\sh{B}=\wt{B}$, where $B$ is an $A$-algebra of finite type \sref[I]{1.9.6.2}, and $\sh{M}=\wt{M}$, where $M$ is a $B$-module \sref[I]{1.1.3.13}; \emph{over the prescheme $X$}, $\OO_X$ is associated to the ring $B$ and $\wt{\sh{M}}$ to the $B$-module $M$; for $\wt{\sh{M}}$ to be of finite type, it is therefore necessary and sufficient for $M$ to be of finite type \sref[I]{1.1.3.13}, hence our assertion.
\end{proof}

\begin{prop}[1.4.6]
\label{2.1.4.6}
Let $Y$ be a prescheme affine over $S$, $X$ and $X'$ two preschemes affine over $Y$ \emph{(hence also over $S$ \sref{2.1.3.5})}.
Let $\sh{B}=\sh{A}(Y)$, $\sh{A}=\sh{A}(X)$, and $\sh{A}'=\sh{A}(X')$.
Then $X\times_Y X'$ is affine over $Y$ \emph{(thus also over $S$)}, and $\sh{A}(X\times_Y X')$ canonically identifies with $\sh{A}\otimes_\sh{B}\sh{A}'$.
\end{prop}

\begin{proof}
\label{proof-2.1.4.6}
By \sref[I]{1.9.6.1}, $\sh{A}\otimes_\sh{B}\sh{A}'$ is a quasi-coherent $\sh{B}$-algebra, thus also a quasi-coherent $\OO_S$-algebra \sref[I]{1.9.6.1}; let $Z$ be the spectrum of $\sh{A}\otimes_\sh{B}\sh{A}'$ \sref{2.1.3.1}.
The canonical $\sh{B}$-homomorphisms $\sh{A}\to\sh{A}\otimes_\sh{B}\sh{A}'$ and $\sh{A}'\to\sh{A}\otimes_\sh{B}\sh{A}'$ correspond \sref{2.1.2.7} to $Y$-morphisms $Z\to X$ and $p':Z\to X'$.
To see that the triple $(Z,p,p')$ is a product $X\times_Y X'$, we can reduce to the case where $S$ is an affine scheme with ring $C$ \sref[I]{1.3.2.6.4}.
But then $Y$, $X$, and $X'$ are affine schemes \sref{2.1.3.4} whose rings $B$, $A$, and $A'$ are $C$-algebras such that $\sh{B}=\wt{B}$, $\sh{A}=\wt{A}$, and $\sh{A}'=\wt{A'}$.
We then know \sref[I]{1.1.3.13} that $\sh{A}\otimes_\sh{B}\sh{A}'$ canonically identifies with the $\OO_S$-algebra $(A\otimes_B A')^\sim$, hence the ring $A(Z)$ identifies with $A\otimes_B A'$ and the morphisms $p$ and $p'$ correspond to the canonical homomorphisms $A\to A\otimes_B A'$ and $A'\to A\otimes_B A'$.
The proposition then follows from \sref[I]{1.3.2.2}.
\end{proof}

\begin{cor}[1.4.7]
\label{2.1.4.7}
Let $\sh{F}$ (resp.~$\sh{F}'$) be a quasi-coherent $\OO_X$-module (resp.~$\OO_{X'}$-module); then $\sh{A}(\sh{F}\otimes_Y\sh{F}')$ canonically identifies with $\sh{A}(\sh{F})\otimes_{\sh{A}(Y)}\sh{A}(\sh{F}')$.
\end{cor}

\begin{proof}
\label{proof-2.1.4.7}
We know that $\sh{F}\otimes_Y\sh{F}'$ is quasi-coherent over $X\times_Y X'$ \sref[I]{1.9.1.2}.
Let $g:Y\to S$, $f:X\to Y$, and $f':X'\to Y$ be the structure morphisms, such that the structure morphism
\oldpage{11}
$h:Z\to S$ is equal to $g\circ f\circ p$ and to $g\circ f'\circ p'$.
We define a canonical homomorphism
\[
  \sh{A}(\sh{F})\otimes_{\sh{A}(Y)}\sh{A}(\sh{F}')\to\sh{A}(\sh{F}\otimes_Y\sh{F}')
\]
in the following way: for every open $U\subset S$, we have canonical homomorphisms $\Gamma(f^{-1}(g^{-1}(U)),\sh{F})\to\Gamma(h^{-1}(U),p^*(\sh{F}))$ and $\Gamma(f^{\prime-1}(g^{-1}(U)),\sh{F}')\to\Gamma(h^{-1}(U),p^{\prime*}(\sh{F}'))$ \sref[0]{0.4.4.3}, thus we obtain a canonical homomorphism
\[
  \Gamma(f^{-1}(g^{-1}(U)),\sh{F})\otimes_{\Gamma(g^{-1}(U),\OO_Y)}\Gamma(f^{\prime-1}(g^{-1}(U)),\sh{F}')\to\Gamma(h^{-1}(U),p^*(\sh{F}))\otimes_{\Gamma(h^{-1}(U),\OO_Z)}\Gamma(h^{-1}(U),p^{\prime*}(\sh{F}')).
\]

To see that we have defined an isomorphism of $\sh{A}(Z)$-modules, we can reduce to the case where $S$ (and as a result $X$, $X'$, $Y$, and $X\times_Y X'$) are affine scheme, and (with the notation of Proposition~\sref{2.1.4.6}), $\sh{F}=\wt{M}$, $\sh{F}'=\wt{M'}$, where $M$ (resp.~$M'$) is an $A$-module (resp.~an $A'$-module).
Then $\sh{F}\otimes_Y\sh{F}'$ identifies with the sheaf on $X\times_Y X'$ associated to the $(A\otimes_B A')$-module $M\otimes_B M'$ \sref[I]{1.9.1.3}, and the corollary follows from the canonical identification of the $\OO_S$-modules $(M\otimes_B M')^\sim$ and $\wt{M}\otimes_\wt{B}\wt{M'}$ (where $M$, $M'$, and $B$ are considered as $C$-modules) (\sref[I]{1.1.3.12} and \sref[I]{1.1.6.3}).
\end{proof}

If we apply Corollary~\sref{2.1.4.7} in particular to the case where $X=Y$ and $\sh{F}'=\OO_{X'}$, then we see that the $\sh{A}'$-module $\sh{A}(f^{\prime*}(\sh{F}))$ identifies with $\sh{A}(\sh{F})\otimes_\sh{B}\sh{A}'$.

\begin{env}[1.4.8]
\label{2.1.4.8}
In particular, when $X=X'=Y$ ($X$ being affine over $S$), we see that if $\sh{F}$ and $\sh{G}$ are two quasi-coherent $\OO_X$-modules, then we have
\[
\label{2.1.4.8.1}
  \sh{A}(\sh{F}\otimes_{\OO_X}\sh{G})=\sh{A}(\sh{F})\otimes_{\sh{A}(X)}\sh{A}(\sh{G})
  \tag{1.4.8.1}
\]
up to canonical functorial isomorphism.
If in addition $\sh{F}$ admits a finite presentation, then it follows from \sref[I]{1.1.6.3} and \sref[I]{1.1.3.12} that
\[
\label{2.1.4.8.2}
  \sh{A}(\shHom_X(\sh{F},\sh{G}))=\shHom_{\sh{A}(X)}(\sh{A}(\sh{F}),\sh{A}(\sh{G}))
  \tag{1.4.8.2}
\]
up to canonical isomorphism.
\end{env}

\begin{rmk}[1.4.9]
\label{2.1.4.9}
If $X$ and $X'$ are two preschemes affine over $S$, then the sum $X\sqcup X'$ is also affine over $S$, as the sum of two affine schemes is an affine scheme.
\end{rmk}

\begin{prop}[1.4.10]
\label{2.1.4.10}
Let $S$ be a prescheme, $\sh{B}$ a quasi-coherent $\OO_S$-algebra, and $X=\Spec(\sh{B})$.
For a quasi-coherent sheaf of ideals $\sh{J}$ of $\sh{B}$, $\wt{\sh{J}}$ is quasi-coherent sheaf of ideals of $\OO_X$, and the closed subprescheme $Y$ of $X$ defined by $\wt{\sh{J}}$ is canonically isomorphic to $\Spec(\sh{B}/\sh{J})$.
\end{prop}

\begin{proof}
\label{proof-2.1.4.10}
It follows immediately from \sref[I]{1.4.2.3} that $Y$ is affine over $S$; by Proposition~\sref{2.1.3.1}, we reduce to the case where $S$ is affine, and the proposition then follows immediately from \sref[I]{1.4.1.2}.
\end{proof}

We can also express the result of Proposition~\sref{2.1.4.10} by saying that if $h:\sh{B}\to\sh{B}'$ is a \emph{surjective} homomorphism of quasi-coherent $\OO_S$-algebras, $\sh{A}(h):\Spec(\sh{B}')\to\Spec(\sh{B})$ is a \emph{closed immersion}.

\begin{prop}[1.4.11]
\label{2.1.4.11}
\oldpage{12}
Let $S$ be a prescheme, $\sh{B}$ a quasi-coherent $\OO_S$-algebra, and $X=\Spec(\sh{B})$.
For every quasi-coherent sheaf of ideals $\sh{K}$ of $\OO_S$, we have (denoting by $f$ the structure morphism $X\to S$) $f^*(\sh{K})\OO_X=(\sh{K}\sh{B})^\sim$ up to canonical isomorphism.
\end{prop}

\begin{proof}
\label{proof-2.1.4.11}
The question being local on $S$, we can reduce to the case where $S=\Spec(A)$ is affine, and in this case the proposition is none other than \sref[I]{1.1.6.9}.
\end{proof}

\subsection{Change of base prescheme}
\label{subsection:change-of-base-prescheme}

\begin{prop}[1.5.1]
\label{2.1.5.1}
Let $X$ be a prescheme affine over $S$.
For every extension $g:S'\to S$ of the base prescheme, $X'=X_{(S')}=X\times_S S'$ is affine over $S'$.
\end{prop}

\begin{proof}
\label{proof-2.1.5.1}
If $f'$ is the projection $X'\to S'$, then it suffices to prove that $f^{\prime-1}(U')$ is an affine open set for every affine open subset $U'$ of $S'$ such that $g(U')$ is contained in an affine open subset $U$ of $S$ \sref{2.1.2.1}; we can thus reduce to the case where $S$ and $S'$ are affine, and it suffices to prove that $X'$ is then an affine scheme \sref{2.1.3.4}.
But then \sref{2.1.3.4} $X$ is an affine scheme, and if $A$, $A'$, and $B$ are the rings of $S$, $S'$, and $X$ respectively, then we know that $X'$ is the affine scheme with ring $A'\otimes_A B$ \sref[I]{1.3.2.2}.
\end{proof}

\begin{cor}[1.5.2]
\label{2.1.5.2}
Under the hypotheses of Proposition~\sref{2.1.5.1}, let $f:X\to S$ be the structure morphism, $f':X'\to S'$ and $g':X'\to X$ the projections, such that the diagram
\[
  \xymatrix{
    X\ar[d] &
    X'\ar[l]_{g'}\ar[d]^{f'}\\
    S &
    S'\ar[l]_g
  }
\]
is commutative.
For every quasi-coherent $\OO_X$-module $\sh{F}$, there exists a canonical isomorphism of $\OO_{S'}$-modules
\[
\label{2.1.5.2.1}
  u:g^*(f_*(\sh{F}))\isoto f_*'(g^{\prime*}(\sh{F})).
  \tag{1.5.2.1}
\]
In particular, there exists a canonical isomorphism from $\sh{A}(X')$ to $g^*(\sh{A}(X))$.
\end{cor}

\begin{proof}
\label{proof-2.1.5.2}
To define $u$, it suffices to define a homomorphism
\[
  v:f_*(\sh{F})\to g_*(f_*'(g^{\prime*}(\sh{F})))=f_*(g_*'(g^{\prime*}(\sh{F})))
\]
and to set $u=v^\sharp$ \sref[0]{0.4.4.3}.
We take $v=f_*(\rho)$, where $\rho$ is the canonical homomorphism $\sh{F}\to g_*'(g^{\prime*}(\sh{F}))$ \sref[0]{0.4.4.3}.
To prove that $u$ is an isomorphism, we can reduce to the case where $S$ and $S'$, hence $X$ and $X'$, are affine; with the notation of Proposition~\sref{2.1.5.1}, we then have $\sh{F}=\wt{M}$, where $M$ is a $B$-module.
We then note immediately that $g^*(f_*(\sh{F}))$ and $f_*'(g^{\prime*}(\sh{F}))$ are both equal to the $\OO_{S'}$-module associated to the $A'$-module $A'\otimes_A M$ (where $M$ is considered as an $A$-module), and that $u$ is the homomorphism associated to the identity (\sref[I]{1.1.6.3}, \sref[I]{1.1.6.5}, \sref[I]{1.1.6.7}).
\end{proof}

\begin{rmk}[1.5.3]
\label{2.1.5.3}
We do not have that Corollary~\sref{2.1.5.2} remains true when $X$ is not assumed affine over $S$, even when $S'=\Spec(\kres(s))$ ($s\in S$) and $S'\to S$ is the canonical morphism \sref[I]{1.2.4.5}---in which case $X'$ is none other than the \emph{fibre $f^{-1}(s)$} \sref[I]{1.3.6.2}.
In other words, when $X$ is not affine over $S$, the operation
\oldpage{13}
``direct image of quasi-coherent sheaves'' does not commute with the operation of ``passing to fibres''.
However, we will see in Chapter~III \sref[III]{3.4.2.4} a result in this sense, of an ``asymptotic'' nature, valid for \emph{coherent} sheaves on $X$ when $f$ is proper~(5.4) and $S$ is Noetherian.
\end{rmk}

\begin{cor}[1.5.4]
\label{2.1.5.4}
For every prescheme $X$ affine over $S$ and every $s\in S$, the fibre $f^{-1}(s)$ (where $f$ denoted the structure morphism $X\to S$) is an affine scheme.
\end{cor}

\begin{proof}
\label{proof-2.1.5.4}
It suffices to apply Proposition~\sref{2.1.5.1} with $S'=\Spec(\kres(s))$ and to use Corollary~\sref{2.1.3.4}.
\end{proof}















